\chapter{Introduction}

\section{Process control laborotory}
The process control laborotory (`The Lab') exists to provide the process control student the opportunity to get their hands dirty and apply theoretical knowledge gained in the classroom to a practical problem.  From the real control valves and measuring equipment to the top-class A/D conversion and historian system, from the hands-on building of equipment to the programming of complex controllers in a simulated environment, the lab gives many oppertunities for exploration.

\section{This manual}
There are many possibilities to explore in the lab but such diversity and scope brings about confusion if not properly mapped out.  This manual represents a collection of the experience of many students.  It is an organic document, which grows with each successful project, whether it is stricty academic or not.  The contents of the manual include, but are not limited to the running of the rigs, settings on the computers and the general hints.  Use it wisely and add to it to let it grow.